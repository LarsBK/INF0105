Hvis et menneske mister en arm, ben eller nyre, kan ingen komme og si at vedkommende ikke lenger er et menneske, og det samme gjelder for mennesker som er f�dt med ulike <<defekter>>. Og hvis man kan miste, eller f�des uten diverse organer som de fleste andre mennesker har, og likevel v�re en del av arten homo sapiens, da vil det vel ikke v�re noe i veien for � erstatte disse s�kallte defektene med menneskeskapte hender, h�r, ben osv? Vi har i veldig lang tid drevet med dette. I dag har briller blitt mer vanlig enn noen gang f�r, og i f�lge Vision Council of America bruker 75 prosent av alle amerikanere en form for synskorreksjon. Briller er selvf�lgelig en mild form for � kompansere for defekter eller skavanker, men det er ogs� eksempler p� blinde som i en viss forstand har f�tt synet igjen, ved hjelp av et kamera, en datamaskin og kobling direkte inn i hjernen. Allerede i 2000 gav et slikt systemet, laget av Dr. Dobelle, syn nok til � navigere, til en tidligere blind mann. Og i 2002 hjalp dette systemet en bonde, mest kjent som Jens, muligheten til � kj�re bil.
Hvor langt kan man g� i denne utbyttingsprosessen f�r man ikke lenger er et menneske?
De fleste vil nok komme fram til at man kan bytte ut det aller meste i kroppen, bortsett fra hjernen, og fremdeles v�re et menneske. Men hva om man kunne bytte ut deler av, eller hele hjernen, og fremdeles v�re den samme personen? 
