Hvis et menneske mister en arm, ben eller nyre, kan ingen komme og si at vedkommende ikke lenger er et menneske, og det samme gjelder for mennesker som er født med ulike <<defekter>>. Og hvis man kan miste, eller fødes uten diverse organer som de fleste andre mennesker har, og likevel vøre en del av arten homo sapiens, da vil det vel ikke være noe i veien for å erstatte disse såkallte defektene med menneskeskapte hender, hår, ben osv.? Vi har i veldig lang tid drevet med dette. I dag har briller blitt mer vanlig enn noen gang før, og i følge Vision Council of America bruker 75 prosent av alle amerikanere en form for synskorreksjon. Briller eller linser er selvfølgelig en mild form for å kompansere for defekter eller skavanker, men det er også eksempler på blinde som i en viss forstand har fått synet igjen, ved hjelp av et kamera, en datamaskin og kobling direkte inn i hjernen. Allerede i 2000 gav et slikt systemet, laget av Dr. Dobelle, syn nok til å navigere, til en tidligere blind mann. Og i 2002 hjalp dette systemet en bonde, mest kjent som Jens, muligheten til å kjøre bil.
Hvor langt kan man gå i denne utbyttingsprosessen før man ikke lenger er et menneske?
De fleste vil nok komme fram til at man kan bytte ut det aller meste i kroppen, bortsett fra hjernen, og fremdeles være et menneske. Men hva om man kunne bytte ut deler av, eller hele hjernen, og fremdeles være den samme personen? 
