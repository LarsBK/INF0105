Futuristen Raymond Kurzweil har skrevet en rekke b�ker der han kommer med forutsigelser
om teknologisk utvikling. Blant de mer kjente er forutsigelsen om at i 1998 ville en datamaskin kunne sl� de beste menneskelige spillerene i sjakk (Deep Blue slo Garry Kasparov i 1997), og i starten av det 21-�rhundret skulle internett f� en eksplosiv �kning i antall brukere og innhold, og vi skulle f� datamaskiner i lomma som kan hente informasjon fra internett tr�dl�st.
Til sammen har idag kun en av hans 108 forutsigelser om teknologisk utvikling, fram til
2009, vist seg � v�re helt gal.
Dette gir oss en pekepinn p� at ogs� hans utsagn om fremtiden, mest sansynlig
har rot i virkeligheten.

I boken The Singularity is Near, antar han at det er mulig � lage en s�kalt
sterk kunstig intelligens, som vil si kunstig intelligens som er like god som,
eller bedre enn menneskelig intelligens og at dette vil skje mellom �r 2015 og 2045.
