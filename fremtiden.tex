Futuristen Raymond Kurzweil har skrevet en rekke bøker der han kommer med forutsigelser
om teknologisk utvikling. Blant de mer kjente er forutsigelsen om at i 1998 ville en datamaskin kunne slå de beste menneskelige spillerene i sjakk (Deep Blue slo Garry Kasparov i 1997), og i starten av det 21-århundret skulle internett få en eksplosiv økning i antall brukere og innhold, og vi skulle få datamaskiner i lomma som kan hente informasjon fra internett trådløst.
Til sammen har idag kun en av hans 108 forutsigelser om teknologisk utvikling, fram til
2009, vist seg å være helt gal.
Dette gir oss en pekepinn på at også hans utsagn om fremtiden, mest sansynlig
har rot i virkeligheten.

I boken The Singularity is Near, antar han at det er mulig å lage en såkalt
sterk kunstig intelligens, som vil si kunstig intelligens som er like god som,
eller bedre enn menneskelig intelligens og at dette vil skje mellom år 2015 og 2045.
