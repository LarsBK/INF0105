Hvordan vi velger å takle dette etiske spørsmålet, som blir mer og mer aktuelt, vil være et resultat av de etiske verdier samfunnet vi tilhører vil inneha i samtiden. Det er sannsynligis ikke få som vil ha et problem med hva som gjør oss til menneskelige vesener, troen på en sjel og at noe uforklarlig skiller oss fra alt annet i verden, vil bli enda mer marginalisert. Og da har vi ikke gått inn på problematikken anngående muligheten til evig liv ved å kunne klone sin egen personlighet.
Hvordan vil student- eller jobbhverdagen være, når man må ta hensyn til maskiner på lik linje med kollegaer? Eller hva med å stemme inn roboter med personstatus i bystyret eller på stortinget? Er dette virkelig en verden vi vil ta del i?