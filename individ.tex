La oss først som sist enkelt skille mellom en person og et menneske: Et menneske er et individ av arten homo sapiens, mens en person må ha egenskaper som selvbevissthet, mulighet til å utføre irrasjonelle handlinger, lære og erfare og så videre. Disse egenskapene, nevnt under person, er alle egenskaper som ikke går på fysiske ting, men heller væremåte eller egenskaper i sinnet. Så hvis disse egenskapene kan tenkes at en gang ikke så langt inne i fremtiden, som er nødvendige for å vøre en person, ikke er avhengige av en kropp, er det da så utenkelig at man kan gå til en bevissthetsbackup, for å lagre bevisstheten sin i ventemodus for sikker hetens skyld? Hva er egentlig forskjellen mellom minnene våre og det som er lagret på en harddisk til en datamaskin?
Hvis vi så godtar at det er personen som er essensen i det å være menneske, og at alle de egenskapene som kreves for å være en person, kan dekkes av ikke-fysiske egenskaper. Da må man også betrakte et "individ" av en ikke-organisk karakter, som et individ, og til og med en person, hvis dette individet oppfyller kravene til å være en person. Alt annet ville være spesiesisme, og urimlig siden man da ikke legger vekt på kun relevante egenskaper. Hvorvidt det en gang i fremtiden, teknologisk vil være mulig, er et spørsmål ingen kan svare sikkert på. Men i Kurzwails The Singularity is Near, antar han at det er mulig å lage en såkalt sterk kunstig intelligens, som vil si en kunstig intelligens som er like god som, eller bedre enn menneskelig intelligens. Han antar at dette vil ha skjedd mellom år 2015 og 2045. Hvorvidt dette er et rimelig tidsestimat, eller i det hele tatt blir mulig, kan bare tiden vise - på samme måte som da Kurzweil i 1990 påsto at det skulle være mulig å koble seg trådløst på internett, og mange av hans andre prognoser.

