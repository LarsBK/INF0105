La oss f�rst som sist enkelt skille mellom en person og et menneske: Et menneske er et individ av arten homo sapiens, mens en person m� ha egenskaper som selvbevissthet, mulighet til � utf�re irrasjonelle handlinger, l�re og erfare og s� videre. Disse egenskapene, nevnt under person, er alle egenskaper som ikke g�r p� fysiske ting, men heller v�rem�te eller egenskaper i sinnet. S� hvis disse egenskapene kan tenkes at en gang ikke s� langt inne i fremtiden, som er n�dvendige for � v�re en person, ikke er avhengige av en kropp, er det da s� utenkelig at man kan g� til en bevissthetsbackup, for � lagre bevisstheten sin i ventemodus for sikker hetens skyld? Hva er egentlig forskjellen mellom minnene v�re og det som er lagret p� en harddisk til en datamaskin?
Hvis vi s� godtar at det er personen som er essensen i det � v�re menneske, og at alle de egenskapene som kreves for � v�re en person, kan dekkes av ikke-fysiske egenskaper. Da m� man ogs� betrakte et "individ" av en ikke-organisk karakter, som et individ, og til og med en person, hvis dette individet oppfyller kravene til � v�re en person. Alt annet ville v�re spesiesisme, og urimlig siden man da ikke legger vekt p� kun relevante egenskaper. Hvorvidt det en gang i fremtiden, teknologisk vil v�re mulig, er et sp�rsm�l ingen kan svare sikkert p�. Men i boken The Singularity is Near, av den kjente futuristen Ray Kurzweil, antar han at det er mulig � lage en s�kalt sterk kunstig intelligens, som vil si kunstig intelligens som er like god som, eller bedre menneskelig intelligens, mener Kurzweil at dette vil skje mellom �r 2015 og 2045. Hvorvidt dette er et rimelig tidsestimat, eller i det hele tatt blir mulig, kan bare tiden vise - p� samme m�te som da Kurzweil i 1990 p�sto at det skulle v�re mulig � koble seg tr�dl�st p� internett, og mange av hans andre prognoser.

